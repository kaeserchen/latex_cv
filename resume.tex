\documentclass{kaeserchen_cv}

% Reduce margin
\geometry{
    a4paper, 
    margin=20mm,
}

\begin{document}
\thispagestyle{empty}
\name{Christine}{Kaeser-Chen}
\subtitle{Software Engineer}{Engineering Manager}
\contact{christine.kaeserchen@gmail.com}{(847)-859-9148}{@kaeserchen}


\section{Motivation}
\begin{flushleft}
I am a software engineer and researcher in computer vision and machine learning.
My main motivation is building things to capture and recreate moments that
matter in life. I am interested in research and applications in the fields of
neural networks, virtual and augmented reality, and computational photography.
\end{flushleft}

\section{Experience}
\begin{flushleft}
\experienceentry
    {Google}
    {Senior Software Engineer, Tech Lead / Manager}
    {2013-present}
    {Mountain View, CA}

I lead a small engineering team in Daydream (Google VR) exploring new ways
for input and interaction in VR. We work at the intersection of computer
vision, machine learning (neural networks in particular), and HCI, in order
to build new input methods to make VR feel even more immersive. \\
\end{flushleft}

Prior to that, I was: \\
\textbullet ~ Founding team member of Google Cardboard. \\
\textbullet ~ Software engineer on the Android team working on new Camera features
              and other core apps. \\

\experienceentry{Flutter, Bot Square Inc.}
                {Engineering Intern}
                {2012-2013}
                {San Franisco, CA}

I worked on the Flutter app, with which users could use hand gestures to control media playback on computers: \\
\textbullet ~ Investigated new computer vision features, and 
              improved machine learning algorithms for gesture detection. \\
\textbullet ~ Developed new user interactions. \\

\experienceentry{Disney Research Z\"urich}
                {Research Intern}
                {2010}
                {Z\"urich, Swizerland}

I worked in the video processing team at Disney Research Zurich: \\
\textbullet ~ Developed an interactive control platform using 
              multi-touch screens for a computational stereoscopic camera system. \\
\textbullet ~ Co-authored the stereoscopic control algorithms for the 
              computational camera system.

\section{Education}
\educationentry{ETH Z\"urich}
               {Master of Science in Computer Science}
               {2009-2012}
               {Z\"urich, Swizerland}
               {Focus Areas: Visual Computing, Theoretical Computer Science \\
                Thesis Topic: \textit{Computational Sports Broadcasting: Automated Director 
                                      Assistance for Live Sports} \\}
\newline
\educationentry{The Hong Kong University of Science and Technology}
               {Bachelor of Science}
               {2005-2009}
               {Hong Kong}
               {Majors: Mathematics, Computer Science}

\section{Publications}
\begin{flushleft}
\begingroup
\renewcommand{\section}[2]{}
\makeatletter
\renewcommand\@biblabel[1]{\textbullet}
\makeatother
\setlength{\bibsep}{2pt}
\bibliography{publications}
\bibliographystyle{ieeetr}
\nocite{*}
\endgroup
\vspace{-0\baselineskip}
\end{flushleft}

\section{Awards and Honors}
\begin{awardssection}
    \award{2015}{Cannes Lions Mobile Grand Prix}{Google Cardboard}
    \award{2009-2012}{Excellence Scholarship \& Opportunity Programme Award}{ETH Z\"urich}
    \award{2009}{Academic Achievement Award, First Class Honor}{HKUST}
\end{awardssection}


\section{Skills}
\begin{skillssection}
    \skill{Programming}{C\texttt{++}, Python, Java}
    \skill{Experience}{TensorFlow, Google VR SDK, Android, OpenCV, OpenGL}
    \skill{Languages}{English, Chinese, German}
\end{skillssection}

\end{document}

